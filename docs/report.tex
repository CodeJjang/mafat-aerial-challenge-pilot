\documentclass[]{article}
\usepackage{graphicx}

%opening
\title{Mafat (DDR\&D) Challenge\\
	Detection and fine grained classification of objects in aerial imagery
	}
\author{Aviad Moreshet\\204311112}

\begin{document}
\date{}
\maketitle

\begin{b}
Background...
\end{b}

\section{The Task}
...

\section{The Dataset}
Most of the dataset does not contain objects at all. Only 51 images are fully tagged. 2161 images are partially tagged, 7121 images were verified to contain no objects at all, and 36 images are untagged.\\
Regarding labels, the dataset definitely biases towards small vehicles, which appear more than the other classes, combined. Distribution can be seen in Figure 1:
\begin{figure}[!h]
\centering
\includegraphics[width=0.7\linewidth]{"charts/Dataset Coarse Grained Labels Distribution"}
\caption{Dataset Coarse Grained Labels Distribution}
\label{fig:Dataset Coarse Grained Labels Distribution}
\end{figure}

As mentioned before, large and small vehicles are further tagged with fine-grained labels.
charts...
charts...

On average, there are 8 objects in an image, while the most crowded image contains 164 objects altogether.


\section{Challenges}
1. Oriented Bounding Box \\
The dataset consisted of Oriented Bounding Boxes (Bounding Boxes which are not axis aligned). \\
All the known formats however supports axis aligned Bounding Boxes only, called Horizontal Bounding Boxes. 
Therefore, I had to wrap each OBB in HBB.
\\\\
2. Classes with a point instead of Bounding Box\\
The utility pole class objects are marked with a point instead of 4 points Bounding Box. Therefore, for the beginning I've decided to remove all the utility poles objects from the dataset (and also images which contained only utility poles, 533 in total), as they are extremely difficult to detect.
\\\\
3. High resolution images\\
Some images did not fit into GPU (94). I decided to ignore them for now, and taking into account images with max resolution of 1000.
\\\\
4. Subclasses\\
I decided to ignore subclasses for now, since most of the detection networks modules do not support them.
\\\\
5. Falsely annotated images\\
Some image annotations contained bounding boxes with coordiantes outside of the image. I decided to crop these coordinates to fit inside image.
\\\\
6. Small objects\\
The objects appeared in the images are much smaller than standard images networks handle. Therefore, the Region Proposal Network missed all the objects.
\section{State Of The Art Computer Vision Networks}
...

\section{Experiments Results}
...

\end{document}